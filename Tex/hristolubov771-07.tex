\documentclass[a4paper,12pt]{article}

\usepackage{cmap}					% поиск в PDF
\usepackage[T2A]{fontenc}			% кодировка
\usepackage[utf8]{inputenc}			% кодировка исходного текста
\usepackage[english,russian]{babel}	% локализация и переносы

\usepackage{amsmath,amsfonts,amssymb,amsthm,mathtools}
\usepackage{icomma}

\usepackage{ amssymb }
\usepackage{ dsfont }
\mathtoolsset{showonlyrefs=true}

\usepackage{euscript}
\usepackage{mathrsfs}
\usepackage{latexsym}

\usepackage{graphicx} 
\usepackage{array}
\usepackage{textcomp}

\renewcommand\thesubsection{\alph{subsection}}
	
\author{Христолюбов Максим, 771}
\title{Домашнее задание 2}
\date{ }

\begin{document} % Конец преамбулы, начало текста.

\maketitle

\section*{Задание 1}
\hspace{0.5cm}
Т.к. ребра идут только от вершин с большим номеров к вершинам с меньшим номером, то из вершины с номером $n$ можно попасть только в вершины с номером $k<n$, значит, в вершину с номером $n$ нельзя попасть из вершины $n$. Так как это верно для всех $n$, то граф - DAG.

\section*{Задание 2}
\hspace{0.5cm}
Очевидно для турниров с $2$ вершинами утверждение верно. Пусть оно верно для турниров с $k$ вершинами, рассмотрим турнир с $k+1$ вершиной. Выберем вершину $v_{0}$. Тогда по нашему предположению в оставшихся $k$ вершинах есть путь длины $k-1$ $v_{1},v_{2},...v_{k}$. Если в последовательности, описывающей путь есть $v_{i},v_{i+1}$ такие, что существует ребро из $v_{i}$ в $v_{0}$ и из $v_{0}$ в $v_{i+1}$, то путь длины $k$ $v_{1},..v_{i},v_{0},v_{i+1},..v_{k}$ - искомый. Если таких $v_{i},v_{i+1}$ не существует, то возможны 2 варианта. Если есть ребро $(v_{1},v_{0})$, то значит все ребра ориентированы в $v_{0}$, следовательно, последовательность $v_{1},..v_{k},v_{0}$ - искомая, если же есть ребро $(v_{0},v_{1})$, то последовательность $v_{0},v_{1},..v_{k}$ - искомая. По индукции получаем требуемое утверждение.

Алгоритм поиска пути $A_{n}$ выглядит так

$A_{n}=waysearch(G_{n})$:

\hspace{0.5cm}	if $n=2$ 
	
	\hspace{1cm}	return: путь из одной вершины графа во вторую
		
\hspace{0.5cm}	Удаляем произвольную вершину $v_{0}$
	
\hspace{0.5cm}	$A_{n-1}=waysearch(G_{n-1})$
	
\hspace{0.5cm}	Из $A_{n-1}$ получаем $A_{n}$ прибавляя к $A_{n-1}$ вершину $v_{0}$ как показано 

\hspace{0.5cm} после предположения индукции.
	
\hspace{0.5cm}	return $A_{n}$
	
В худшем случае на каждом $k$-ом шаге придется перебирать все $k-1$ вершин на существования ребра из них в $v_{0}$ и делать фиксированное кол-во операций. Всего рекурсивных вызовов $n-2$, $\sum\limits^{n}_{k=3}{k-2}=\Theta(n^2)$ операций

\section*{Задание 3}
\hspace{0.5cm}
a) Прямым ребром является ребро графа, соединяющее предка и потомка в дереве, но не входящего в него. Значит вторая вершина $B$ ребра $e$ должны быть в поддереве первой вершины $A$, т.е. время $d_{A}<d_{B}$ и $f_{B}<f_{A}$, но $A$ - не родитель $B$. Проверить выполнение данных условий можно за $\Theta(1)$

б) Перекрестное ребро - ребро графа, которое соединяет вершины, которые не являются для друг друга предком и потомком. Значит, $f_{A}<d_{B}$ или $f_{B}<d_{A}$

\section*{Задание 4}
\hspace{0.5cm}
Воспользуемся алгоритмом поиска сильных компонент связности, они и будут этими областями. Для этого проведем топологическую сортировку, проведя поиск в глубину и расположив вершины в порядке убывания закрытия вершин. В сопряженном графе проведем поиск в глубину по вершинам в порядке топологической сортировки. Каждое дерево будет компонентой сильной связности, так как из всех вершин, в которые можно попасть из начальной, так же можно попасть в начальную, поскольку начальная находится левее в топологической сортировке сопряженного (изначального) графа. Ни в какие же лишние попасть не получится, так как из  вершины можно попасть только в лишнюю вершину, которая находится левее в топологической сортировке сопряженного графа, но во всех них мы уже побывали. Таким образом, полученные компоненты сильной связности - искомые области.

\section*{Задание 5}
\hspace{0.5cm}
Будем обходить лабиринт поиском в глубину и искать выход, если после обхода всех комнат лабиринта (вершин графа) выхода не будет, значит его нет. Для того чтобы не запутаться по каким коридорам (ребрам) уже проходили, при переходе в другую комнату у входа в коридор будем класть $k$ монеток, где $k$ - номер комнаты (нумерация с $1$ - начальной комнаты, в каждой новой посещенной комнате увеличивается на $1$), если мы по нему сейчас пойдем. Так же при входе в $k+1$ комнату у того прохода из которого пришли кладем $k$ монет. Это необходимо чтобы различать ребра по которым мы пришли в вершину и ушли. Очевидно, что по коридорам, помеченным монеткой не надо ходить. Если мы пришли в комнату, а в ней у одного из проходов уже есть монетки, значит мы здесь уже были и надо вернуться назад, не оставляя никаких монет в этой комнате. Если все коридоры, выходящие из комнаты были пройдены, то возвращаемся по коридору с наименьшим кол-во монет. Таким образом реализуется поиск выхода в глубину, так как по каждому коридору мы пройдем не более 2 раз, то за $O(m)$ переходов получится найти выход.

\section*{Задание 6}
\hspace{0.5cm}
Если добавленное ребро соединяет предка со своим потомком, то оно не влияет на связность графа. Если же ребро соединяет потомка $r$ с предком $l$, то появляется цикл из вершин $(r-l)$. Если добавленны 2 ребра $v$ и $u$ ($l_{v}\leq l_{u}$ для определенности), причем $l_{u}\leq r_{v}$, то группа вершин $(l_{v}-max(r_{v},r_{u}))$ - компонента сильной связности. Связность такого графа можно изобразить графически на прямой, на которой отмечены вершины $a_{1},...a_{n}$, а добавленные ребра - это отрезки $[r; l]$. Тогда если отрезки пересекаются, то их объединение образует компоненту сильной связности. Задача сводится к нахождению отдельных объединений отрезков, после нанесения на прямую $m$ отрезков.

Отсортируем добавленные ребра (отрезки) по возрастанию $l_{k}$, тогда если $l_{k+1}\leq r_{k}$, то объединим отрезки в один $[l_{k}; max(r_{v},r_{u})]$. Будем сравнивать полученный отрезок с $k+2$, пока не получим, что $l_{p+1}\geq r_{p}$, где $r_{p}$ - правый конец отрезка полученного на предыдущем шаге. Т.к. отрезки отсортированны по возрастанию это будет значит конец цельного объединение отрезков, которое является компонентой сильной связности. Перейдем к следующему отрезку, найдем следующую компоненту сильной связности. Будем повторять эти действия начиная с $k=1$, пока массив вершин не закончится. Полученные объединения $[b_{p}; c_{p}]$ - искомые компоненты связности из вершин от $b_{p}$-ой до $c_{p}$-ой. 

На быструю сортировку добавленных ребер уйдет $\Theta(m\log{m})$ операций, на объединение отрезков $\Theta(m)$ операций, всего - $\Theta(m\log{m})$.

\section*{Задание 7}
\hspace{0.5cm}
Исходный граф состоит из компонент связности. В каждой компоненте связности существует эйлеровый цикл, поскольку степень каждой вершины четна. Построим эйлеровый цикл по алгоритму поиска эйлерова цикла (лекция).

Запустим на графе поиск в глубину, начиная с какой-то вершины, определяющую первую компоненту связности, причем будем заносить в 2 массива $L_{k}$ и $R_{k}$ посещенные вершины из $L$ и $R$ и считать их количество. Когда вершины, достижимые из той, откуда начался поиск в глубину, закончатся все вершины из этой компоненты связности. В ней ровно $min(\textbar L_{k}\textbar,\textbar R_{k}\textbar)$ ребер из паросочетания. Действительно, так как по компоненте связности построен эйлеров цикл, то ребра, соединяющие вершины из $V=min( L_{k}, R_{k})$ c следующими за ними вершинам в эйлеровом цикле, будут входить в паросочетание. Докажем, что оно максимально. Так как все вершины из $V=min( L_{k}, R_{k})$ уже задействованы, то больше ребер с незадействованными вершинами нет, значит, это паросочетание - максимально. 

Проведем поиска максимального паросочетания по остальным компонентам связности, просто беря еще не пройденную алгоритмом поиска в глубину вершину. Тогда максимальным паросочетанием для графа будет обьединение максимальных паросочетаний его компонент связности.






\end{document} % Конец текста.