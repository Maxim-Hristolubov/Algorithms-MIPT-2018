\documentclass[a4paper,12pt]{article}

\usepackage{cmap}					% поиск в PDF
\usepackage[T2A]{fontenc}			% кодировка
\usepackage[utf8]{inputenc}			% кодировка исходного текста
\usepackage[english,russian]{babel}	% локализация и переносы

\usepackage{amsmath,amsfonts,amssymb,amsthm,mathtools}
\usepackage{icomma}

\usepackage{ amssymb }
\usepackage{ dsfont }
\mathtoolsset{showonlyrefs=true}

\usepackage{euscript}
\usepackage{mathrsfs}
\usepackage{latexsym}

\usepackage{graphicx} 
\usepackage{array}
\usepackage{textcomp}

\renewcommand\thesubsection{\alph{subsection}}
	
\author{Христолюбов Максим, 771}
\title{Домашнее задание 2}
\date{ }

\begin{document} % Конец преамбулы, начало текста.

\maketitle

\section*{Задание 1}  
\hspace{0.5cm}

При вычитании делимость чисел на их НОД не меняется, значит, НОД всех чисел, записанных изначально на доске, и в конце одинаковый. Следовательно, так все числа будут равны, то их НОД будут равен им самим, т. е. на доске будет записан НОД всех этих чисел.

\section*{Задание 2}  
\hspace{0.5cm}

Найдем НОД по алгоритму Евклида. НОК находится из равенств НОК$\cdot$ НОД$=a\cdot b$. Сложность алгоритма $O(n^3)$, т к на умножение требуется $O(n^{\log_{2}^{3}})$ по алгоритму быстрого умножения, на деления $O(n^2)$, значит, сложность алгоритма нахождения НОК - $O(n^3)$
 
\section*{Задание 3}  
\hspace{0.5cm}

 $\sum {a_{i}\cdot a_{j}} = \frac{1}{2}\cdot \left( \left( \sum a_{i}\right)^2-\sum{a_{i}^2}\right)$
 	
Для нахождение суммы элементов требуется $O(n)$ операций, возведение в квадрат одного числа, равного суме элементов - константа, для суммы квадратов - $O(n)$, т к возведение элемента в квадрат происходит за константу действий. Значит $\sum {a_{i}\cdot a_{j}}$ вычисляется за $O(n)$ операций.

\section*{Задание 4}  
\hspace{0.5cm}
а) $a=36$, $b=6$, $f(n)=n^2=n^{\log_{6}{36}}=\Theta (n^{log_{b}{a}})$

По мастер-теореме $T(n)=\Theta (n^2\log{n})$

б) $a=3$, $b=3$, $f(n)=n^2=n^{\log_{3}{3}+1}=\Omega(n^{\log_{b}{a}+\epsilon})$, где $\epsilon=\frac{1}{2}$

По мастер-теореме $T(n)=\Theta(f(n))=\Theta(n^2)$

в) $a=4$, $b=2$, $f(n)=\frac{n}{\log{n}}=O(n^{\log_{2}{4}-\frac{1}{2}})=O(n^{\log_{b}{a}-\epsilon})$, $\epsilon=\frac{1}{2}$

По мастер-теореме $T(n)=\Theta(n^{\log_{b}{a}})=\Theta(n^2)$

\section*{Задание 5}  
\hspace{0.5cm}
$T(n)=n\cdot T(\frac{n}{2})+O(n)=n\left( \frac{n}{2}\cdot T\left(\frac{n}{4}\right) +O(n)\right) +O(n)=\frac{n^2}{2}\cdot T(\frac{n}{4})+O(n^2)=\frac{n^3}{8}\cdot T(\frac{n}{8})+O(n^3)=...=\frac{n^k}{2^{\frac{k(k-1)}{2}}}\cdot T(\frac{n}{2^k})+O(n^k)$, $k=\log_{2}{n}$

$T(n)=\frac{n^{\log_{2}{n}}}{\left( 2^{\log_{2}{n}}\right)^{\frac{\log_{2}{n}-1}{2}}} \cdot T(1)+O(n^{log_{2}{n}})=n^{\log_{2}{n}-\frac{1}{2}\log_{2}{n}+\frac{1}{2}}+O(n^{log_{2}{n}})=\ =n^{\frac{1}{2}\log_{2}{n+\frac{1}{2}}}+O(n^{log_{2}{n}}) = O(n^{\log_{2}{n}}\sqrt{n})=\Omega(n^{\log_{2}{n}})$

\section*{Задание 6}  
\hspace{0.5cm}
a) Пусть для определенности $\alpha<1-\alpha$

$T(n)\leq T(\alpha n)+T((1-\alpha)n)+\Theta(n)=T(\alpha^2n)+2T(\alpha(1-\alpha)n)+T((1-\ -\alpha)^2 n)+2\Theta(n)\leq...\leq2^{\log_{\alpha}{n}}T(1)+\log_{\alpha}{n}\cdot\Theta(n)=Cn+\Theta(n\log_{\alpha}{n})=O(n\log{n})$

$T(n)\geq 2^{\log{1-\alpha}{n}}T(1)+\log_{1-\alpha}{n}\Theta(n)=Cn+\Theta(n\log_{1-\alpha}{n})=\Omega(n\log{n})$

$T(n)=\Theta(n\log{n})$

б) $T(n)=T(\frac{n}{2})+2T(\frac{n}{4})+\Theta(n)=T(\frac{n}{4})+2T(\frac{n}{8})+4T(\frac{n}{16})+2\Theta(n)\leq...\leq2^{\log{2}{n}}T(1)+\log_{2}{n}\Theta(n)=O(n\log{n})$

$T(n)\geq2^{\log{4}{n}}T(1)+\log_{4}{n}\Theta(n)=\Omega(n\log{n})$

$T(n)=\Theta(n\log{n})$

в) $T(n)=27T(\frac{n}{3})+\frac{n^3}{\log^2{n}}=3^{3\cdot 2}T(\frac{n}{3^2})+3^3\cdot\frac{\left( \frac{n}{3}\right) ^3}{\log^2{\frac{n}{3}}}+\frac{n^3}{\log^2{n}}=3^{3\cdot 2}T(\frac{n}{3^2})+\frac{n ^3}{\log^2{\frac{n}{3}}}+\frac{n^3}{\log^2{n}}=3^{3\cdot 3}T\left(\frac{n}{3^3} \right) +\frac{n^3}{\log^2{\frac{n}{3^2}}}+\frac{n^3}{\log^2{\frac{n}{3}}}+\frac{n^3}{\log^2{n}}=3^{3\cdot \log_{3}{n}}T(1)+n^3\sum\limits_{k=0}^{\log_{3}{n}-1}\frac{1}{\log^2{\frac{n}{3^k}}}=n^3T(1)+n^3\sum\limits_{k=0}^{\log_{3}{n}-1}\frac{1}{\log^2{\frac{n}{3^k}}}=\Omega(n^3)$

Т.к. бесконечный ряд обратных квадратов натуральных чисел сходится, т.е. имеет конечную сумму, а $\log^2{\frac{n}{3^k}}$ - некоторые натуральные числа, причем не равные друг другу и упорядоченные, то $\sum\limits_{k=0}^{\log_{3}{n}-1}\frac{1}{\log^2{\frac{n}{3^k}}}$ меньше чем конечная сумма обратных квадратов натуральных чисел, значит, меньше бесконечной суммы
меньше некоторой константы.

$T(n)=O(n^3+n^3\cdot C)=O(n^3)=\Theta(n^3)$

\section*{Задание 7}  
\hspace{0.5cm}
$(i!)^{-1}(\mod p)=x$

$x\cdot i!\equiv 1(\mod p)$

$x\cdot i!+p\cdot k=1$

Это уравнение решается в целых числах с помощью алгоритма Евклида за константное кол-во $C_{k}$ арифметических действий  для конкретного $x_{k}$. Решим это уравнение для $i\in{1..p-1}$ и получим $x_{1}$,$x_{2}$,..,$x_{i}$. Это можно сделать за $\Theta(n)$ арифметических действий, т.к. сумма действий ограничена сверху $C_{k}n$, где  $C_{k}$-самый большой коэффициент, а снизу - $C_{j}\frac{n}{2}$, где $C_{j}$- медиана массивов коэффициентов.

$(i!)^{-1}(\mod p)=(1\cdot2\cdot...\cdot p\cdot...\cdot i)^{-1}(\mod p)=1^{-1}\cdot2^{-1}\cdot...\cdot (i)^{-1}(\mod p)= x_{1}x_{2}...x_{i}(\mod p)$

$x_{i}$ перемножаются за $\Theta(n)$ действий, а делится по модулю за константу. Значит можно найти $x_{1}=1^{-1}$ по алгоритму Евклида, записать в массив, умножить его на $2^{-1}$ и получить $x_{2}$? записать в массив и тп - на это потребуется $n$ шагов, на каждом - константное кол-во действий - всего $\Theta(n)$.

Следовательно весь алгоритм работает за $\Theta(n)$ и вычисляет $(i!)^{-1}(\mod p)$.



\section*{Задание 7 с семинара}  
\hspace{0.5cm}

$N_{k}$ - длина числа в $k$-ой записи.

$n=N_{2}\leq N_{10}\leq N_{16}=4\cdot N_{2}=O(n)$, значит, $N_{10}=\Theta(n)$, где $n=N_{2}=\Theta(n)$

Умножение числа $N$ длины $n$ на число $M$. На первом шаге:

$N=M+N_{1}$,где $N_{1}<M$

$N\cdot M=M^2+M*N_{1}$

Найдем $M\cdot N_{1}$ тем же алгоритмом или же если одно из них достаточно маленькое просто умножим "столбиком" за константное время, и будем продолжать, пока $N_{i}$ или $M_{i}$ не станет равно $0$.

Корректность. На каждом шаге или $N_{i}$, или $M_{i}$ уменьшается хотябы на $1$, значит настанет момент когда один из них станет равен $0$.

Если одно из чисел достаточно маленькое, то его можно считать константой, а умножение числа длины $n$ на константу происходит за $\Theta(n)$. Если считать что второе число достаточно большое, то это значит, что числа отличаются на небольшое число шагов, и кол-во действий которое надо произвести - константа, так как на каждом шаге $\Theta(n)$ операций, то всего потребуется $\Theta(n)$.


\end{document} % Конец текста.