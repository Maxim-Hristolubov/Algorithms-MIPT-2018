\documentclass[a4paper,12pt]{article}

\usepackage{cmap}					% поиск в PDF
\usepackage[T2A]{fontenc}			% кодировка
\usepackage[utf8]{inputenc}			% кодировка исходного текста
\usepackage[english,russian]{babel}	% локализация и переносы

\usepackage{amsmath,amsfonts,amssymb,amsthm,mathtools}
\usepackage{icomma}

\usepackage{ amssymb }
\usepackage{ dsfont }
\mathtoolsset{showonlyrefs=true}

\usepackage{euscript}
\usepackage{mathrsfs}
\usepackage{latexsym}

\usepackage{graphicx} 
\usepackage{array}
\usepackage{textcomp}

\renewcommand\thesubsection{\alph{subsection}}
	
\author{Христолюбов Максим, 771}
\title{Домашнее задание 2}
\date{ }

\begin{document} % Конец преамбулы, начало текста.

\maketitle

\section*{Задание 6 с семинара}  
\hspace{0.5cm}
Нет, не возможно. Допустим алгоритм проверил все биты кроме хотябы одного. В худшем случае или слева от не проверенного бита будет стоять $0$, либо справа $1$. Тогда для случая, когда непроверенный бит - $1$ и для случая с $0$ алгоритм получил одинаковый набор данных, а поскольку алгоритм детерминирован, то он всегда будет давать один ответ, т.е. ошибаться в одном из случаев.

\section*{Задание 7 с семинара}  
\hspace{0.5cm}
a) Пронумеруем монеты от 1 до 10. Сравним группы из монет 1, 2, 3 и 4, 5, 6.

Если весы покажут равенство, тогда фальшивая монета в остальных четырех. Сравним 7, 8, 9 и заведомо настоящие 1, 2, 3. 

Если весы покажут равенство, то фальшивая монета - последняя. 

Если нет весы покажут, что 7, 8, 9 тяжелее значит, фальшивая монета тяжелее остальных. Сравним 7 и 8, в случае равенства 9 - фальшивая,  иначе фальшивая более тяжелая из 7 и 8.

Если весы покажут, что 1, 2, 3 тяжелее чем 4, 5, 6, то будем делать следующие шаги. В случае когда они будут легче из симметричности потребуется произвести те же действия, поменяв кучки местами.

Сравним 1, 4 и 2, 3. 

Если весы показали равенство, то фальшивая монета среди 5, 6, причем она легче остальных. Сделаем последнее сравнение и найдем фальшивку. 

Если весы показали, что 1, 4 тяжелее, то сравним 1 и заведомо настоящей 7. Если они равны, то 4 - фальшивая, иначе 1 - фальшивка. 

Если показали что 2, 3 тяжелее, то фальшивая - самая тяжелая из них.

\section*{Задание 1}
\hspace{0.5cm}
Можно отсортировать слова сначала по последнему символу устойчивой сортировкой. Это можно сделать за $\Theta(n)$, так как кол-во латинских символов константа. Занумеруем буквы, запишем в элементы массива С сумму количеств каждого из символов, номер которого меньше чем номер элемента в массиве С, а потом пройдемся по словам и поставим на $C_{p}$ место $p$-ый символ, $C_{p}=C_{p}+1$ за $O(n)$. То есть воспользуемся RadixSort, потребуется $k$ сортировок по $\Theta(n)$ операций - $\Theta(nk)$.

\section*{Задание 2}
\hspace{0.5cm}
1) Спросим сначала чему равен $\frac{n}{3}$ и $\frac{2n}{3}$ элементы. Если $\frac{2n}{3}$ больше, то это значит что $t$-ый элемент не может лежать в первой трети массива - левее $\frac{n}{3}$-ого элемента, если $\frac{n}{3}$ больше, то искомый элемент не лежит в последней трети массива. Повторим операцию с частью массива длины $\frac{2n}{3}$, где может находится искомый элемент. На каждом шаге массив уменьшается в $\frac{2}{3}$ раз, значит за $\log_{\frac{3}{2}}{n}$ шагов найдется искомый элемент, сложность - $\Theta(\log{n})$.

\section*{Задание 3}
\hspace{0.5cm}
Разделим монеты на $3$ части, по крайней мере $2$ из которых содержат $\lfloor\frac{n}{3}\rfloor$ кол-во монет. Сравним на весах эти 2 равные по кол-ву монет части. Если они равны, то фальшивя монета находится в остальных монетах, если одна куча перевесила, то фальшивая монета находится в более легкой куче. Повторим операцию для той кучи, где находится фальшивая монета. На каждом кол-во монет уменьшается в $\frac{2}{3}$ раз, следовательно за $\log_{3}{n}+с$ взвешиваний мы дойдем до кучи из одной или $2$ монет, в которой можно найти фальшивую за $0$ или $1$ взвешивание. Значит, можно сделать $\log_{3}{n}+с$ операций и найти фальшивую монету.

\section*{Задание 4}
\hspace{0.5cm}
Одно взвешивание дает информацию о принадлежности фальшивой монеты одному из $3$ множеств ( множества зависят от того какие имеено кучи монет взвешивать). Для того чтобы алгоритм мог определить фальшивую монету, ни для каких $2$ монет результаты взвешиваний не должны быть одинаковыми, иначе в силу детерминированности алгоритма он будет ошибаться в одном из случаев. Запишем результаты взвешиваний как троичные числа, в которых $k$-ый разряд будет говорить в какой из $3$ частей лежит монета. Тогда для каждых $2$ различных монет им соответствующие троичные числа должны быть разными. Троичных чисел длины $k$ - $3^{k}\geq n$, значит $k\geq\log_{3}{n}$, следовательно требуется произвести не менее $\log_{3}{n}$ взвешиваний.

\section*{Задание 5}
\hspace{0.5cm} 
Сравним элементы, лежащие на $\frac{n}{2}$ местах в массивах - это медианы, поскольку массивы упорядочены. Пусть для определенности $A[\frac{n}{2}]>B[\frac{n}{2}]$(если $A[\frac{n}{2}]=B[\frac{n}{2}]$, то это и есть медиана, она найдена). Тогда общая медиана лежит в первой половине массива $A$ или во второй половине массива $B$, так как значение общей медианы лежит между значениями медиан массивов. Возьмем за исходные массивы соответствующие половины и повторим операцию. На каждом шаге область поиска медианы уменьшается в $2$ раза, значит через $\log_{2}{n}$ шагов останется один элемент - общая медиана. На каждом шаге делается фиксированное кол-во шагов, сложность $\Theta(\log{n})$.

\section*{Задание 6}
\hspace{0.5cm} 
Будем перебирать числа, начиная с $1$ и считать значение, пока значение многочлена не будет больше или равно $y$. В первом случае такого числа $x$ не существует, во втором - число $x$, на котором остановился алгоритм - искомое. Сложение $n$ чисел займет $\Theta(n)$ действий, многочлен станет больше или равен $y$ через $\Theta(\sqrt[n]{y})$ шагов, сложность алгоритма - $\Theta(n\sqrt[n]{y})$.

\section*{Задание 7}
\hspace{0.5cm}
1) Разобьем монеты на пары и сравним их попарно - $\frac{n}{2}$ взвешиваний, те что окажутся тяжелее отложим в первую кучу, те что легче - во вторую. Очевидно самая тяжелая находится в первой, а самая легкая - во второй. Теперь взвесим попарно монеты из первой кучи ($\frac{n}{4}$ взвешиваний) и отбросим легкие, а с тяжелыми повторим то же самое - будем попарно сравнивать и отбрасывать легкие пока не останется единственная монета - самая тяжелая. Кол-во действий будет равно $\frac{n}{4}+\frac{n}{8}+...+\frac{n}{p}\leq\frac{n}{2}$. Со второй кучей будем делать то же самое, только отбрасывать тяжелые и в итоге получим самую легкую, аналогично сделав не более $\frac{n}{2}$ сравнений. Всего $\frac{3n}{2}+с$ сравнений, константа $c$ возникает из-за того что не всегда монеты будут делится ровно пополам.


\end{document} % Конец текста.