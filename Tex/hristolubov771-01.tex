\documentclass[a4paper,12pt]{article}

\usepackage{cmap}					% поиск в PDF
\usepackage[T2A]{fontenc}			% кодировка
\usepackage[utf8]{inputenc}			% кодировка исходного текста
\usepackage[english,russian]{babel}	% локализация и переносы

\usepackage{amsmath,amsfonts,amssymb,amsthm,mathtools}
\usepackage{icomma}

\usepackage{ amssymb }
\usepackage{ dsfont }
\mathtoolsset{showonlyrefs=true}

\usepackage{euscript}
\usepackage{mathrsfs}
\usepackage{latexsym}

\usepackage{graphicx} 
\usepackage{array}
\usepackage{textcomp}

\renewcommand\thesubsection{\alph{subsection}}
	
\author{Христолюбов Максим, 771}
\title{Домашнее задание 1}
\date{ }

\begin{document} % Конец преамбулы, начало текста.

\maketitle

\section*{Задание 5[Шень 1.3.2]}  
\hspace{0.5cm}
Фиксируем $y[j], j=1$ и перебираем $x[i]$, начиная с $i=1$, пока не будет выполнятся $i=k_{1}, x[k_{1}]=y[1]$.

Далее перебираем $x[i]$, начиная с $i=k_{1}+1$, пока не будет выполнятся $i=k_{2}, x[k_{2}]=y[j], j=2$

Повторяем эти действия для всех $y[1]$, либо пока не будет выполнятся равенство $i=n$.
Если по окончании действия цикла, если $j=k+1$ (то есть для всех $y[j]$ до $i=k$ включительно нашлись равные им $x[i]$), тогда $y[j]~-~$подпоследовательность $x[i]$, иначе $~-~$ нет.

Каждый $x[i]$ сравнивается один раз, счетчики $i,j$ пробегают от $1$ до $n$ и $k$ соответственно, значит кол-во операций линейно зависит от $n$ и $k$, и сложность алгоритма $O(n+k)$.	

\section*{Задание 1}
\hspace{0.5cm}
а) Да, верно, т. к. $n \leq n\log_{}n\ \forall n\in \mathds{N}$.

б) Да, верно. Рассмотрим предел $\lim\limits_{n\to\infty}\frac{n\ln{n}}{n^{1+\varepsilon}}=\lim\limits_{n\to\infty}\frac{\ln{n}}{n^{\varepsilon}}=\lim\limits_{n\to\infty}\frac{\frac{1}{n}}{\varepsilon n^{\varepsilon-1}}=\lim\limits_{n\to\infty}\frac{1}{n^{\varepsilon}}=0$ по правилу Лопиталя, значит, $n\ln(n)=O\left( n^{1+\varepsilon}\right) $

\section*{Задание 2}
\subsection*{1.}
\hspace{0.5cm}
a) Да, возможно, например, $g(n)=\frac{n}{\log{n}}, f(n)=n^{2}, h(n)=\frac{n^{2}\log{n}}{n}=n\log{n}=\Theta(n\log{n})$

б) Нет, не возможно.

$\lim\limits_{n\to\infty}\frac{f(n)}{g(n)n^{3}}=\lim\limits_{n\to\infty}\frac{f(n)}{n^{2}}\cdot\lim\limits_{n\to\infty}\frac{1}{g(n)}\cdot\lim\limits_{n\to\infty}\frac{1}{n}=0\cdot0\cdot0=0$. Эти пределы равны $0$ из того, что $f(n) = O\left( n^{2}\right) , g(n) = \Omega(1)$

\subsection*{2.}
\hspace{0.5cm}
$h(n)=O\left( n^{2}\right) $, при $g(n)=1$, $f(n)=n^{2}$

$h(n)=\Omega\left( \frac{1}{n^{C}}\right) \ \forall C>0$, при $g(n)=1, f(n)=\frac{1}{n^{C}}$

\section*{Задание 3}
a) При вводе нового элемента добавляем $1$ в счетчик общего числа элементов ($O(n)$ операций), и прибавляем значение нового элемента к хранящемуся в памяти будущей сумме элементов, которая изначально равна $0$ ($O(n)$ операций). После вводе всех чисел частное суммы и общего числа элементов будет ответом.

б) При вводе нового числа оно сравнивается с значением $max$ (изначально равно первому элементу). Если оно равно ему, то добавляем $1$ ($O(n)$ операций) в счетчик элементов равных максимальному (изначально счетчик $1$), если оно больше $max$, то сбрасываем счетчик ($O(n)$ операций) и записываем значение элемента в $max$ ($O(n)$ операций), если оно меньше - просто переходим к следующему элементу. Значение счетчика - ответ.

в) Запоминаем предыдущий элемент. Если новый элемент равен ему, то добавляем $1$ ($O(n)$ операций) в счетчик последних равных элементов идущих подряд $i$ (изначально счетчик $i=1$). После этого сравниваем $i$ c $max$ длинной наибольшей серии ($O(n)$ операций), если $i>max$, то $max=i$. Если новый элемент не равен предыдущему, то счетчик $i$ сбрасывается. $max$ - ответ.

\section*{Задание 4}
\hspace{0.5cm}
Будем проходить по 3 массивам с помощью 3 индексов - $i,j,k$, изначально равные $1$, тогда $x[1], y[1], z[1]$ - первые элементы массивов.

Сначала сравним между собой $x[1], y[1], z[1]$, и запишем наименьший из них в переменную $p$, увеличим счетчик различных чисел на $1$ (изначально $0$). Будем увеличивать на 1 каждый из индексов $i,j,k$, пока не будут выполнятся неравенства $x[1]>p, y[1]>p, z[1]>p$.

Тогда еще раз сравним $x[1], y[1], z[1]$, и запишем наименьший из них в переменную $p$, увеличим счетчик различных чисел на $1$. Будем продолжать эти действия пока $i,j,k$ не станут равняться кол-ву элементов в соответствующих массивах.

Счетчик различных чисел в итоге даст ответ.

В алгоритме индексы $i,j,k$ пробегают все элементы своих массивов по одному разу, сравнивая их на каждом шаге с $p$ и иногда сравнивая $x[1], y[1], z[1]$, записывая наименьший из них в переменную $p$ и увеличивая счетчик различных чисел на $1$. Кол-во всех этих действий линейно зависит от размеров массивов, значит сложность алгоритма $O(n)$.

Т.к счетчик срабатывает всегда когда изменяется $p$, а $p$ принимает всевозможные значения из массивов без повторений (в силу упорядоченности массивов), то алгоритм работает корректно.

\section*{Задание 5}
\subsection*{a)}
\hspace{0.5cm}
Будем составлять массив $p[i]$ следующим образом: на $i$-том месте в массиве будет стоять длина наибольшей возрастающей подпоследовательности чисел исходного массива, заканчивающейся $i$-ым элементом. Элементы $p[i]$ будут находится так: если перед $i$-тым элементом нет чисел меньше его, то $p[i]=1$, т.~к. единственная возрастающая подпоследовательность состоит из одного $i$-ого числа исходного массива. Если существуют числа меньшие его, то $p[i]=max(p[k_{j}])+1$, где $k_{j}$- номера элементов, которые меньше $i$-ого элемента.

Действительно, подпоследовательности, оканчивающиеся на $i$-ый элемент - это подпоследовательности, в которых перед $i$-ым числом стоит число, меньшее его, длины p[k] с дописанным в конце $i$-ым числом.

Для того чтобы заполнить $p[i]$ необходим сравнить $i$-ый элемент со всеми предыдущими, что займет $O(n)$, длина массива $p[i]$ - $n$, значит сложность алгоритма - $O\left( n^2\right) )$.








\end{document} % Конец текста.