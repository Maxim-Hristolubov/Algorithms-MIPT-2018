\documentclass[a4paper,12pt]{article}

\usepackage{cmap}					% поиск в PDF
\usepackage[T2A]{fontenc}			% кодировка
\usepackage[utf8]{inputenc}			% кодировка исходного текста
\usepackage[english,russian]{babel}	% локализация и переносы

\usepackage{amsmath,amsfonts,amssymb,amsthm,mathtools}
\usepackage{icomma}

\usepackage{ amssymb }
\usepackage{ dsfont }
\mathtoolsset{showonlyrefs=true}

\usepackage{euscript}
\usepackage{mathrsfs}
\usepackage{latexsym}

\usepackage{graphicx} 
\usepackage{array}
\usepackage{textcomp}

\renewcommand\thesubsection{\alph{subsection}}
	
\author{Христолюбов Максим, 771}
\title{Домашнее задание 2}
\date{ }

\begin{document} % Конец преамбулы, начало текста.

\maketitle

\section*{Задание 1}
\hspace{0.5cm}
Добавление элемента в стек реализуется через 2 очереди следующим образом: добавляем элемент в $1$-ую очередь. Извлечение элемента: перекладываем все данные кроме последнего из $1$-ой очереди во $2$-ую и извлекаем этот единственный последний элемент из $1$-очереди. После этого перекладываем все элементы из $2$-ой очереди в первую (или же теперь работаем со $2$-ой как с $1$-ой).

\section*{Задание 2}
\hspace{0.5cm}
Троичное дерево можно хранить построчно, то есть сначала стоит корень дерева, потом его потомки слева на право, потом их потомки, причем сначала иду потомки самого левого элемента-родителя, потом среднего, потом правого. Потомки элемента $A[i]$ - $A[3i], A[3i+1], A[3i+2]$. Родитель элемента $A[i]$ - $A\lfloor\frac{i}{3}\rfloor$.

\section*{Задание 3}
\hspace{0.5cm}
$x$- элемент, $y$ - следующий за ним по возрастанию в дереве. Т.к. у $x$ нет правого потомка, а $y>x$, то $y$ был добавлен в дерево раньше $x$, так как в другом случае он должен был бы быть в правом поддереве $x$. Рассмотрим процесс добавления $x$ в дерево с $y$. В тех узлах где $у$ отправлялся в левое поддерево $x$ тоже будет отправляться в левое поддерево. Так как в дереве нет $z$, такого что $y>z>x$, то в тех узлах где $у$ отправлялся в правое поддерево $x$ тоже будет отправляться в правое поддерево, так как если бы было иначе, то получилось бы что узловой элемент $u>x$, но $y<u$, то есть такое $z=u$ $y>z>x$. Значит, $x$ пройдет по дереву тем же путем что $y$ и отправится в его левое поддерево. После этого $x$ не может ни в каком узле отправится в левое поддерево, так как тогда этот узловой элемент $v$ $y>v>x$, что не возможно. Значит, $x$ дойдет до низа дерева и там запишется, следовательно, $y$ является самым нижним предком x, чей левый дочерний узел так же является предком $x$ или самим $x$.

\section*{Задание 4}
\hspace{0.5cm}
Если бы последующая за $b$ вершина $c$ имела бы левого потомка, тогда бы в случае, когда $c$ находится в поддереве (а именно в правом поддереве) $b$, тогда левый потомок $b$ $d$ был бы $d<b$, $d>a$, что противоречит условию. Если $b$ находится в левом поддереве $c$, то правый потомок $b$ $q$ такой, что $q>b$, но $q<c$ - противоречие. Если же $c$ и $b$ не в поддеревьях друг друга, то это значит, что существует узел $e$ в котором $c$ ушло в его правое поддерево, а $b$ в левое, тогда, если у $c$ есть левый потомок $d$, то $d>e>b$, а $c>d>b$, что противоречит условию. Следовательно у $c$ нет левого потомка.

В случае когда $b$ находится в правом поддереве $a$ левый потомок $b$ $w$ такой, что $w<b$ и $w>a$ - противоречие.Если $a$ в левом поддереве $b$, то если существует правый потомок $a$ $s$, то $s>a$ и $s<b$, что невозможно. Если они находятся не в поддеревьях друг друга, то существует узел $x$, в котором $a$ ушло в его левое поддерево, а $b$ в правое, тогда, если существует правый потомок $a$ $s$, то $s>a$ и $s<b$, что противоречит условию задачи.
Следовательно у $a$ нет правого потомка.


\section*{Задание 5}
\hspace{0.5cm}
а) Если для того, чтобы узнать принадлежит ли $x$ $A$ достаточно выполнить $t=1$ запросов то сделать это можно единственным образом - если запрос будет единственный запрос - принадлежит ли $x$ $A$. Т.е. в таблице должно быть $n$ строчек, каждая из которых говорила бы принадлежит ли $x$ $A$. $s(m,k,1)=n$. Меньше строчек не может быть, так как для каждого элемента алгоритм должен быть способен ответить на вопрос, для этого при $t=1$ каждому элементу должен соответствовать собственный вопрос.

б) Интуиция подсказывает, что $\log{n}$, но как доказать это я не знаю.

\section*{Задание 6}
\hspace{0.5cm}
Если обслуживать клиентов в порядке $k_{1}$, $k_{2}$...$k_{n}$, то время обслуживания будет равно $\sum\limits_{i=1}^{n}\sum\limits_{j=1}^{i}t_{k_{j}}=\sum\limits_{i=1}^{n}\left( (n+1-i)t_{k_{i}}\right) =n(n+1)-\sum\limits_{i=1}^{n}i t_{k_{i}}$. $\sum\limits_{i=1}^{n}i t_{k_{i}}$ будет максимальна если $t_{k_{i}}$ будут идти по возрастанию. Значит, можно отсортировать клиентов быстрой сортировкой за $\Theta(nlog{n})$ и получить последовательность клиентов с минимальным суммарным временем  ожидания клиентов.



\end{document} % Конец текста.