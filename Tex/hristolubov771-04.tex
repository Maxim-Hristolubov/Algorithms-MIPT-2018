\documentclass[a4paper,12pt]{article}

\usepackage{cmap}					% поиск в PDF
\usepackage[T2A]{fontenc}			% кодировка
\usepackage[utf8]{inputenc}			% кодировка исходного текста
\usepackage[english,russian]{babel}	% локализация и переносы

\usepackage{amsmath,amsfonts,amssymb,amsthm,mathtools}
\usepackage{icomma}

\usepackage{ amssymb }
\usepackage{ dsfont }
\mathtoolsset{showonlyrefs=true}

\usepackage{euscript}
\usepackage{mathrsfs}
\usepackage{latexsym}

\usepackage{graphicx} 
\usepackage{array}
\usepackage{textcomp}

\renewcommand\thesubsection{\alph{subsection}}
	
\author{Христолюбов Максим, 771}
\title{Домашнее задание 2}
\date{ }

\begin{document} % Конец преамбулы, начало текста.

\maketitle

\section*{Задание 6 (Итеративный MergeSort)}  
\hspace{0.5cm}
Пусть для простоты в исходном массиве $A$ $2^k$ элементов.На исходный массив можно взглянуть как на массив упорядоченных массивов длины $1$. Изначально $i=1$. Следующий цикл выполняется до тех пока не будет выполнятся $l=2^k$.

Производим сортировку слиянием попарно массивов с началом в $A[2^{l}p]$ и $A[2^{i}p+2^{i-1}]$ длины $l=2^{i-1}$ (на первом шаге каждый из них состоит из одного элемента $l=1$), $p=0,1,2,...,(2^{k-i}-1)$. Получим отсортированные массивы $A[2^{i}p]$ длины $l=2^i$, $p=0,1,...,(2^{k-i-1}-1)$.

На каждой следующей итерации $i=i+1$, т.е. уже полученные отсортированные массивы в новом обозначении будут выглядеть так: массивы с началом в $A[2^{i-1}p]$ длины $l=2^{i-1}$, $p=0,1,...,(2^{k-i}-1)$, что абсолютно соответствует входным данным в первой итерации. Будем продолжать итерации пока не настанет момент $i=k$. В итоге получим отсортированный массив $A[0]$ длины $l=2^k$. Если в $A$ кол-во элементов - не степень $2$, то в последнем подмассиве будет меньше чем $2^{i-1}$ перед слиянием, что не мешает алгоритму.

\section*{Задание 1}
\hspace{0.5cm}
$F(3,5)=\left( (1\cdot3)^2\right) ^2\cdot 3 = 3^5=243$


Это функция быстрого возведения $x$ в степень $m$. Действительно, как было показано в одном из предыдущих заданий для того чтобы быстро возвести в степень нужно разложить $m$ так, чтобы было максимальное кол-во умножений на 2 (возведение $x^k$ в квадрат) и минимальное кол-во добавлений 1 (умножение $x^k$ на $x$). Это разложение можно получить из двоичной записи $m$. Начинаем с $1$, если в разряде стоит $1$, то нужно возвести число в квадрат и умножить на $x$, что и делается, т.к $1$ переводится в $SX$, что приводит к возведению в квадрат и умножению, а если в разряде $0$, то просто возвести в квадрат, т.е.  $S$.

Так как на каждом шаге выполняется возведение в квадрат и в худшем случае еще умножение, а всего шагов - двоичная запись числа $m$, то сложность алгоритма $\Theta(\log m)$.

\section*{Задание 2}
\hspace{0.5cm}
Найдем $\frac{2n}{3}$ и $\frac{2n}{3}-1$ порядковую статистику $r_{i}$ - их номер $k$ и $p$. Точки принадлежащие ровно $\frac{2n}{3}$ отрезкам находятся между $r_{p}$ и $r_{k}$, а также между $l_{k}$ и $l_{p}$, так как если упорядочить $r_{i}$, то внутри отрезка с концом в $r_{k}$ ($\frac{2n}{3}$ порядковой статистики) лежит $\frac{2n}{3}-1$ вложенных отрезков. Эти 2 порядковые статистики можно найти за линейное время, сложность алгоритма $\Theta(n)$.
  
\section*{Задание 3}
\hspace{0.5cm}
$T(n)=T(\frac{n}{7})+T(\frac{5n}{7})+Cn$, так как на следующем шаге рассматривается массив длины не больше $\frac{5n}{7}$, поскольку искомая порядковая статистика находится правее(и левее) чем $4\cdot\frac{n}{2\cdot 7}$ элементов.

$T(n)=O(n)$, докажем по индукции: $T(n)=T(\frac{n}{7})+T(\frac{5n}{7})+Cn= O(\frac{n}{7})+O(\frac{5n}{7})+Cn= O(n)$.

\section*{Задание 4}
\hspace{0.5cm}
Перебираем все элементы массивы по порядку, если элемент равен $0$, то записываем его в конец будущего отсортированного массива (изначально он пуст), после этого проходим по данному в условии массиву еще раз и если элемент равен $1$, то записываем его в конец результирующего массива.


\section*{Задание 5}
\hspace{0.5cm}
$ax+b\equiv 0(\mod M)$

$ax+kM=-b$

Решить данное уравнение можно за $O(n^3)$ по алгоритму евклида (если производить деление за $O(n^2)$). После этого мы будем знать $x$. $n^3$ - полином.


\section*{Задание 6}
\hspace{0.5cm}
В худшем случае массив будет разбиваться каждый раз опорным элементом на часть равную остальному массиву и пустую часть (опорный элемент при partition будет вставать в конец или начало) и потребуется $n$ рекурсивных вызовов.

Для того чтобы в худшем случае глубина стека было $\Theta(\log n)$ за опорный элемент можно брать медиану, тогда массив всегда будет делится на 2 почти равные части.









\end{document} % Конец текста.